% IEEE Paper Template for US-LETTER Page Size (V1)
% Sample Conference Paper using IEEE LaTeX style file for US-LETTER pagesize.
% Copyright (C) 2006-2008 Causal Productions Pty Ltd.
% Permission is granted to distribute and revise this file provided that
% this header remains intact.
%
% REVISION HISTORY
% 20080211 changed some space characters in the title-author block
%
\documentclass[10pt,conference,letterpaper]{IEEEtran}
\usepackage{times,amsmath,epsfig}
\usepackage{algorithm}
\usepackage{algorithmic}
\usepackage{epstopdf}
\usepackage{graphicx}
\renewcommand{\algorithmicrequire}{\textbf{Input:}} 
\renewcommand{\algorithmicensure}{\textbf{Output:}}
\usepackage[tight,footnotesize]{subfigure}
\usepackage{amsfonts}
\usepackage{xspace}
\newcommand{\tabincell}[2]{\begin{tabular}{@{}#1@{}}#2\end{tabular}}  
\newcommand{\frname}{GTS\xspace }
\newcommand{\idxname}{GTIDX\xspace }



%
\title{An Efficient GPU-accelerated Framework for Answering Trajectory Queries}
%
\author{%
% author names are typeset in 11pt, which is the default size in the author block
{Bowen Zhang{\small $~$}, Yanmin Zhu{\small $~$}, Yanyan Shen{\small $~$} }%
% add some space between author names and affils
\vspace{1.6mm}\\
\fontsize{10}{10}\selectfont\itshape
% 20080211 CAUSAL PRODUCTIONS
% separate superscript on following line from affiliation using narrow space
\,Department of Computer Science and Engeneering, Shanghai Jiao Tong University\\
Shanghai, China 200240\\
\fontsize{9}{9}\selectfont\ttfamily\upshape
%
% 20080211 CAUSAL PRODUCTIONS
% in the following email addresses, separate the superscript from the email address 
% using a narrow space \,
% the reason is that Acrobat Reader has an option to auto-detect urls and email
% addresses, and make them 'hot'.  Without a narrow space, the superscript is included
% in the email address and corrupts it.
% Also, removed ~ from pre-superscript since it does not seem to serve any purpose
\{zbw0046,yzhu,shenyy\}@sjtu.edu.cn\\
% add some space between email and affil
\vspace{1.2mm}\\
\fontsize{10}{10}\selectfont\rmfamily\itshape
% 20080211 CAUSAL PRODUCTIONS
% separated superscript on following line from affiliation using narrow space \,
%$^{*}$\,Second Company\\
%Address Including Country Name\\
%\fontsize{9}{9}\selectfont\ttfamily\upshape
%% 20080211 CAUSAL PRODUCTIONS
%% removed ~ from pre-superscript since it does not seem to serve any purpose
%$^{2}$\,second.author@second.com
}
%
\begin{document}
\maketitle
%
\begin{abstract} 
As the development of smart devices equipped with GPS, here comes a large amount of trajectory data, implying many useful information about our daily life. This calls for a trajectory analytics framework able to process mass of various kinds of queries efficiently. GPU, which has been widely equipped in data centers, can accelerate queries by handling them in parallel. However, existing GPU-accelerated trajectory storage systems are optimized for specific kind of query, suffering from the problem of efficiency when they are used for processing queries they are not optimized for. To solve this problem, we propose a framework optimized for the features of GPU which supports both two basic kinds of queries for big trajectory data. We design a unified storage component with an index called \idxname with a cell-based trajectory storage, which combines and links quadtree, grid and trajectories together to support pruning methods for two basic kinds of queries. Based on the storage component, to make full use of the parallel power of GPU, the query processing in our framework is optimized for the issues of GPU computing including load-balancing, coalesce memory accessing and less data transferring. We implement our framework and evaluate it on two real-life trajectory datasets, which shows our framework is able to conduct two basic types of queries on large scale trajectory data efficiently. Moreover, our framework achieves a speedup of 38x for range query and 67x for top-k similarity query than the implementation on CPU, demonstrating that our framework really achieves the goal of accelerating both two kinds of queries on large-scale trajectory data by GPU.
\end{abstract}

%
%
%\bibliographystyle{IEEEtran}
%
%\bibliography{IEEEabrv,IEEEexample}

\end{document}

